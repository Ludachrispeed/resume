\documentclass[12pt]{article}

\usepackage[T1]{fontenc}         % https://texfaq.org/FAQ-why-inp-font
\usepackage{fontawesome}         % icons
\usepackage{geometry}            % page dimensions
\usepackage{multicol}            % multi-column environments
\usepackage{textcomp}            % bullet symbol
\usepackage{tcolorbox}           % color boxes using tikz
\usepackage[explicit]{titlesec}  % formats for section headings
\usepackage{xcolor}              % color control, like green!20!white
\usepackage{hyperref}            % hyperlinks; this should be last

% Page dimensions
\geometry{
  left=0.6in,
  right=0.6in,
  top=0.4in,
  bottom=0.4in
}

% Define "\CC" as a command to typeset "C++" nicely
\def\CC{{C\nolinebreak[4]\hspace{-.05em}\raisebox{.4ex}{\tiny\bf ++}}}

% Bullet style in bulleted lists
\renewcommand{\labelitemi}{\textopenbullet}

% Amount of space before paragraphs
\parskip 5pt

% Amount of space before and after a multicols environment
\multicolsep 5pt

% Don't indent the first line of a paragraph
\parindent 0pt

% No page numbers
\pagestyle{empty}

% Hyperlink configuration
\hypersetup{
  colorlinks=true,
  urlcolor=blue!80!black,
}

% Section header style (this behavior comes from the titlesec package)
\titleformat{\section} % the sectioning command to be redefined
            [hang] % shape
            {\sffamily\color{red}} % format to be applied
            {\thesection} % label
            {0pt} % sep
            {\Large #1} % before code
            [\titlerule{}] % after code

% Subsection header style (from titlesec)
\titleformat{\subsection} % the sectioning command to be redefined
            [hang] % shape
            {\large\bfseries} % format to be applied
            {\thesubsection} % label
            {0pt} % sep
            {\begin{tcolorbox}[nobeforeafter]#1\end{tcolorbox}}  % before code
            [] % after code

% Section spacing (moves sections 15pts into the left margin)
\titlespacing*{\section} % the sectioning command
              {-15pt} % left
              {0pt} % before-sep
              {5pt} % after-sep
              [0pt] % right-sep

% Subsection spacing
\titlespacing*{\subsection} % the sectioning command
              {0pt} % left
              {10pt} % before-sep
              {2pt} % after-sep
              [0pt] % right-sep

% Remove page number at bottom of page
\thispagestyle{empty}

% Adjust spacing in enumerate environment
\makeatletter
\AtBeginDocument {
  \renewcommand{\@listi} {
    \setlength{\itemsep}{0pt}
    \setlength{\parsep}{3pt}
    \setlength{\partopsep}{5pt}
    \setlength{\parskip}{0pt}
    \setlength{\topsep}{0pt}
    \setlength{\leftmargin}{30pt}
  }
  \renewcommand{\@listii} {
    \setlength{\itemsep}{5pt}
    \setlength{\topsep}{0pt}
  }
}
\makeatother

% Tag style (the green pillboxes used to list technologies)
\newtcbox{\tag}{
  on line,
  arc=7pt,
  boxrule=1pt,
  boxsep=0pt,
  top=2pt,
  bottom=2pt,
  left=6pt,
  right=6pt,
  before upper={\rule[-2pt]{0pt}{12pt}},
  colframe=green!50!black,
  colback=green!20!white}

%%%%%%%%%%%%%%%%%%%%%%%%%%%%%%%%%% Document %%%%%%%%%%%%%%%%%%%%%%%%%%%%%%%%%%%%

\begin{document}

\begin{center}
  {\sffamily\huge\textbf{Chris Clark}}
\end{center}

\tcbset{topbar/.style={
    frame empty,
    box align=top,
    width=(\linewidth-2mm)/2,
    nobeforeafter}}

% Note that "topbar" is a self-defined style defined with tcbset
\begin{tcolorbox}[
    topbar,
    boxsep=0pt,
    left=1pt,
    coltext=black!60,
    colback=white,
    halign=left,
    fontupper=\itshape\bfseries\large]

  I have a passion for programming, designing systems, and learning with people.
  I strive for simplicity and elegance. I help teams be productive through
  clarity, vision, and teamwork.

\end{tcolorbox}
\begin{tcolorbox}[
    topbar,
    left=1pt,
    colback=yellow!25!white]
  \begin{tabular}{r l}
    \faEnvelope\      & \texttt{cfclrk@gmail.com} \\
    \faPhone\         & \texttt{571-732-9827} \\
    \faGithub\        & \url{http://github.com/cfclrk} \\
    \faStackOverflow\ & \url{http://stackoverflow.com/u/340613} \\
    \faLinkedin\      & \url{http://linkedin.com/in/cfclrk} \\
  \end{tabular}
\end{tcolorbox}

% Section: Skills
% ------------------------------------------------------------------------------

\section*{Skills}

\begin{multicols}{2}
  \raggedright

  \subsection*{Most Recent and Familiar}

  \textbf{Languages}: Python, Bash, Lisp

  \textbf{Other}: Kubernetes, Git, Docker, AWS, Azure, Terraform

  \subsection*{Worked with in the Past}

  \textbf{Languages}: Go, Javascript, Haskell, Rust, Java, \LaTeX, C, \CC,
  MATLAB/Octave, CSS/LESS/SASS, SQL

  \textbf{Other}: Ansible, Packer, Webpack

  \vfill
  \columnbreak

\end{multicols}

% Section: Experience
% ------------------------------------------------------------------------------

\section*{Experience}

\subsection*{Cloud Engineer at IronNet \hfill March 2019 -- Present}

\begin{itemize}
\item I am mostly working on automation for creating and managing Kubernetes
  clusters (and the applications running on the clusters)
\item Help debug and fix Kubernetes or application deployment issues
\item Help create and implement a company cloud strategy; for example, how to
  create, manage, and secure AWS (and Azure) accounts, and how to audit
  important activity in those accounts.
\end{itemize}

\tag{Python} \tag{Bash} \tag{Kubernetes} \tag{helm} \tag{Rancher} \tag{AWS}
\tag{Azure} \tag{Terraform}

\subsection*{QA Engineer at Fugue \hfill Jul 2015 -- Jan 2019}

The QA team was in charge of designing and implementing a test harness for Fugue
(which was a platform for defining, creating, and monitoring infrastructure in
AWS and Azure). I created much of this system, as well as a variety of
general-purpose internal tooling.

\begin{itemize}
\item Helped design and create an end-to-end test harness for Fugue (Python,
  Bash, Go)
\item Wrote a utility for clearing AWS accounts (Python)
\item In charge of test VMs in AWS and Azure (Ansible, Packer, Go, Azure
  Resource Manager)
\item Created and maintained many continuous-integration builds in Travis and
  Concourse
\item Devised a literate-programming approach for running and documenting tests
\end{itemize}

\tag{Python} \tag{Bash} \tag{Clojure} \tag{Go} \tag{Haskell} \tag{AWS}
\tag{Azure} \tag{Ansible} \tag{Packer}

\subsection*{Web Developer at Wells Fargo \hfill Feb 2015 -- Jul 2015}

Member of a small team creating an internal auditing and task-tracking
application.

\tag{Java} \tag{Spring} \tag{Hibernate} \tag{YUI} \tag{Javascript}

\subsection*{Startup Entrepreneur \hfill May 2014 -- Jan 2015}

Primarily focused on two business ventures:

\begin{itemize}
\item Partnered with a small team of entrepreneurs creating an application for
  pairing high-level executives with new opportunities.

\item Working with a medical clinic to create a multi-user web application that
  maintained an audit trail of communications between doctors, patients, nurses,
  and call centers.
\end{itemize}

\tag{MEAN stack} \tag{Docker} \tag{Java 8} \tag{Scala} \tag{MySQL} \tag{Bash}
\tag{\LaTeX}

\subsection*{Consultant at Red Hat \hfill Aug 2011 -- May 2014}

As a consultant, I traveled to client sites and worked on a variety of projects
that Red Hat was involved with. Mostly, this involved integrating the JBoss
suite of products which Red Hat had recently acquired.

\begin{itemize}
\item \textit{GEICO} -- Helped to build and expand web services. Also helped to
  create an internal error tracking and aggregation platform using Grails and
  AngularJS.

\item \textit{Cigna} -- Worked on an application for inferring customer
  information using a rules engine.

\item \textit{Kohl's} -- Sole developer porting the main website to a website
  for kiosk machines.

\end{itemize}

\tag{Java} \tag{JBoss} \tag{JBoss Rules} \tag{jBPM} \tag{Hibernate}
\tag{MyBatis} \tag{Grails} \tag{AngularJS}

% Section: Education
% ------------------------------------------------------------------------------

\section*{Education}

\textsc{University of Virginia}, School of Engineering and Applied Science,
Charlottesville, VA \\
\textit{B.S. in Computer Science, May 2011}

% Section: Outside of Work
% ------------------------------------------------------------------------------

\section*{Outside of Work}

I enjoy family time, quiet time, reading, learning, and doing any kind of
sports.

\begin{itemize}
\item I like playing around with Elisp and Emacs.
\item I created and maintain \url{http://lenaclarklegal.com} for my wife's legal
  practice
\item President of my HOA
\item I love doing any kind of sports
\end{itemize}

\end{document}
