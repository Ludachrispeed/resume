% Uses the "article" documentclass. Sections are redefined using the titlesec
% package (giving them a margin farther left than the other text, and giving
% them a horizontal rule under the section title)

%%%%%%%%%%%%%%%%%%%%%%%%%%%%%%%%%% Preamble %%%%%%%%%%%%%%%%%%%%%%%%%%%%%%%%%%%

\documentclass[a4paper,11pt]{article}

\usepackage{textcomp}       % bullet symbol
\usepackage{titlesec}       % defining formats for section headings
\usepackage{multicol}       % multi-column environments
\usepackage{hyperref}       % hyperlinks
\usepackage[
  left=0.5in,
  right=0.5in,
  top=0.4in,
  bottom=0.4in]{geometry}   % page dimensions

% TODO: I'm not using these sweet tags, but I'm saving the code it case I do
% sometime. They don't really look good yet.
%\usepackage{tikz}
%\usetikzlibrary{positioning}

% defines '\CC' as a command to render 'C++' nicely
\newcommand{\CC}{C\nolinebreak\hspace{-.05em}\raisebox{.4ex}{\tiny\bf
    +}\nolinebreak\hspace{-.10em}\raisebox{.4ex}{\tiny\bf +}}
\def\CC{{C\nolinebreak[4]\hspace{-.05em}\raisebox{.4ex}{\tiny\bf ++}}}

% sets the kind of bullet to use in bulleted lists
\renewcommand{\labelitemi}{\textopenbullet}

% amount of space before paragraphs
\parskip 5pt

% amount of space before and after a multicols environment
\multicolsep 5pt

% don't indent the first line of a paragraph
\parindent 0pt

% configure how hyperlinks look
\hypersetup{
  colorlinks=true,
  linkcolor=red
}

% set the format for sections (this behavior comes from the titlesec package)
\titleformat{\section} % the sectioning command to be redefined
            [hang] % shape
            {\sffamily} % format to be applied
            {\thesection} % label
            {0pt} % sep
            {\Large} % before code
            [\titlerule{}] % after code

% set the format for subsections (from titlesec)
\titleformat{\subsection} % the sectioning command to be redefined
            [hang] % shape
            {\bfseries} % format to be applied
            {\thesubsection} % label
            {0pt} % sep
            {} % before code
            [] % after code

% defines spacing for sections (moves sections 15pts into the left margin)
\titlespacing*{\section} % the sectioning command
              {-15pt} % left
              {0pt} % before-sep
              {5pt} % after-sep
              [0pt] % right-sep

% defines spacing for subsections
\titlespacing*{\subsection} % the sectioning command
              {0pt} % left
              {0pt} % before-sep
              {2pt} % after-sep
              [0pt] % right-sep

% remove page number at bottom of page
\thispagestyle{empty}

% adjust spacing in enumerate environment
\makeatletter
\AtBeginDocument {
  \renewcommand{\@listi} {
    \setlength{\itemsep}{0pt}
    \setlength{\parsep}{3pt}
    \setlength{\partopsep}{0pt}
    \setlength{\parskip}{0pt}
    \setlength{\topsep}{0pt}
    \setlength{\leftmargin}{15pt}
  }
  \renewcommand{\@listii} {
    \setlength{\itemsep}{5pt}
    \setlength{\topsep}{0pt}
  }
}
\makeatother

%%%%%%%%%%%%%%%%%%%%%%%%%%%%%%% Document Text %%%%%%%%%%%%%%%%%%%%%%%%%%%%%%%%%%

\begin{document}

% note: not using the \maketitle command. We make our own title our own way.
\begin{center}
  {\sffamily \huge \textbf{Chris Clark}} \\
  Frederick, MD \enspace\textopenbullet\enspace
  https://github.com/Ludachrispeed \enspace\textopenbullet\enspace
  ludachrispeed@gmail.com
\end{center}

% --------------------------- SECTION: EDUCATION -------------------------------

\vspace{-9pt}
\section*{Education}

  \textsc{University of Virginia}, School of Engineering and Applied Science,
  Charlottesville, VA \\
  \textit{B.S. in Computer Science, May 2011}

%  \subsection*{Activities}
%
%  \textit{RideForward}: A UVA-based group of students and professors converting
%  gas-powered cars to electric drive. \\
%  \textit{Positions held}: Technical team member, business team member, webmaster

  \begin{multicols}{3}
    \raggedright

    \begin{itemize}
    \item \textit{Math}: Calculus II, III, Discrete Math, Differential Equations, Probability,
      Linear Algebra
    \item Computer Networks
    \item Operating Systems
    \item Digital Logic Design
    \item Computer Architecture
    \item Theory of Computation
    \item Algorithms
    \item Computer Vision
    \end{itemize}
  \end{multicols}

% ------------------------ SECTION: RELEVENT SKILLS ----------------------------

  \section*{Relevent Skills}

  \begin{multicols}{2}
    \raggedright

    \subsection*{Programming Languages}

    Python, Go, Javascript, Java, Bash, Fish, SQL, \LaTeX, C, \CC, MATLAB/Octave, Gnuplot, Groovy

    \subsection*{Dev Ops}

    Jenkins, Ansible, Docker, Vagrant, Git, Subversion
    \vfill
    \columnbreak

    \subsection*{Web Technologies}

    \textsl{Backend}: Spring (Framework, MVC, Security), Hibernate, AWS, jBPM, Play Framework,
    Grails, XML, relational databases, node.js, npm

    \textsl{Frontend}: HTML5, angularJS, CSS, less, SCSS, bower, grunt, gulp


  \end{multicols}

% -------------------------- SECTION: EXPERIENCE -------------------------------

\vspace{-9pt}
\section*{Experience}

\subsection*{Automation Engineer at Fugue}
\vskip -20pt
\hfill \textit{August 2015 -- Present}

Developed test harness and tooling for Fugue.

\subsection*{Web Developer at Wells Fargo}
\vskip -20pt
\hfill \textit{February 2015 -- August 2015}

Member of a small team creating an internal auditing and task-tracking application.

Technologies: Java, Spring, Hibernate, YUI, Javascript

\subsection*{Startup Entrepreneur and Chief Software Developer/Architect}
\vskip -20pt
\hfill \textit{May 2014 -- Present}

\begin{itemize}

  \vskip 5pt
  \item \textit{CXOzone}: Partnered with a small team of entrepreneurs to create a web application
    pairing high-level executives with new opportunities. See \texttt{http://cxo.zone}.

  \item \textit{CarrotMD}: Multi-user web application which maintains an audit trail of
    communications between doctors, patients, nurses, and call centers. Built for small medical
    clinics. More info upon request.

\end{itemize}

Technologies: node.js, express, AngularJS, SailsJS, Docker, MongoDB, Java 8, Spring, Hibernate,
MySQL, bash, \LaTeX

\subsection*{Consultant for Red Hat Consulting}

  \vskip 5pt
  \begin{itemize}

    \item \textit{Website Development at GEICO \hfill Aug 2012 -- May 2014}

      Maintained RESTful web services ultimately consumed by mobile devices. Also helped to develop
      an internal metrics-tracking and error-tracking application using Grails and AngularJS.

      Technologies: Java 6, Spring MVC, JBoss AS, JAXB, jBPM, XForms, Hibernate, MyBatis, Grails,
      AngularJS

%  maybe one day I'll use these sweet tags? Still needs more work.
%
%      \begin{tikzpicture}[tag/.style={fill=blue!20,draw=blue!80,rounded corners}]
%        \node (a) [tag] {Java 6};
%        \node (b) [tag, right=of a] {Spring};
%        \node (c) [tag, right=of b] {JAXB};
%        \node (d) [tag, right=of c] {XForms};
%        \node (e) [tag, right=of d] {JAXB};
%        \node (f) [tag, right=of e] {MyBatis};
%        \node (g) [tag, right=of f] {Grails};
%        \node (h) [tag, right=of g] {AngularJS};
%      \end{tikzpicture}

    \item \textit{Website development at Cigna \hfill Nov 2011 -- July 2012}

      Member of a two-person team working on an internal application for viewing customer
      information. Also introduced and presented programming best practices to other software
      developers.

      Technologies: Java 6, JBoss Drools, Spring Framework, iBATIS

    \item \textit{Website development at Kohl's \hfill Aug 2011 -- Nov 2011}

      Sole developer in charge of porting functionality from the desktop website to a website
      running on kiosk machines. Implemented new UI features as requested by a UI requirements team.

      Technologies: Java 4, Starteam, IntelliJ, JSP

  \end{itemize}


% ---------------------- SECTION: OTHER SKILLS/HOBBIES -------------------------

\vskip 4pt
\section*{Other Skills/Hobbies}
\vskip 5pt

  \begin{itemize}
    \item Designed, created, and maintain \texttt{http://lenaclarklegal.com} for my wife's legal
      practice (deployed on AWS).
    \item UVA Varsity Cross Country, Track \& Field \hfill \textit{2006--2008}
  \end{itemize}

\end{document}
