% Resume uses 'article' documentclass. Sections are redefined using the titlesec
% package (giving them a margin farther left than the other text, and giving
% them a horizontal rule under the section title)

%%%%%%%%%%%%%%%%%%%%%%%%%%%%%%%%%% Preamble %%%%%%%%%%%%%%%%%%%%%%%%%%%%%%%%%%%

\documentclass[a4paper,11pt]{article}

\usepackage{textcomp} % for bullet symbol
\usepackage{titlesec} % for defining formats for section headings
\usepackage{multicol} % for multi-column environments
\usepackage[left=0.3in, right=0.4in, top=0.4in, bottom=0.5in]
           {geometry} % for page dimensions

% TODO: I'm not using these sweet tags, but I'm saving the code it case I do
% sometime. Also they're not really done.
%\usepackage{tikz}
%\usetikzlibrary{positioning}

% defines '\CC' as a command to render 'C++' nicely
\newcommand{\CC}{C\nolinebreak\hspace{-.05em}\raisebox{.4ex}{\tiny\bf
    +}\nolinebreak\hspace{-.10em}\raisebox{.4ex}{\tiny\bf +}}
\def\CC{{C\nolinebreak[4]\hspace{-.05em}\raisebox{.4ex}{\tiny\bf ++}}}

% sets the kind of bullet to use in bulleted lists
\renewcommand{\labelitemi}{\textopenbullet}

% amount of space before paragraphs
\parskip 5pt

% amount of space before and after a multicols environment
\multicolsep 5pt

% don't indent the first line of a paragraph
\parindent 0pt

% set the format for sections (this behavior comes from the titlesec package)
\titleformat{\section} % the sectioning command to be redefined
            [hang] % shape
            {\sffamily} % format to be applied
            {\thesection} % label
            {0pt} % sep
            {\Large} % before code
            [\titlerule{}] % after code

% set the format for subsections (from titlesec)
\titleformat{\subsection} % the sectioning command to be redefined
            [hang] % shape
            {\bfseries\sffamily} % format to be applied
            {\thesubsection} % label
            {0pt} % sep
            {} % before code
            [] % after code

% defines spacing for sections (moves sections 15pts into the left margin)
\titlespacing*{\section} % the sectioning command
              {-15pt} % left
              {0pt} % before-sep
              {5pt} % after-sep
              [0pt] % right-sep

% defines spacing for subsections
\titlespacing*{\subsection} % the sectioning command
              {0pt} % left
              {0pt} % before-sep
              {2pt} % after-sep
              [0pt] % right-sep

% remove page number at bottom of page
\thispagestyle{empty}

% adjust spacing in enumerate environment
\makeatletter
\AtBeginDocument {
  \renewcommand{\@listi} {
    \setlength{\itemsep}{0pt}
    \setlength{\parsep}{3pt}
    \setlength{\partopsep}{0pt}
    \setlength{\parskip}{0pt}
    \setlength{\topsep}{0pt}
    \setlength{\leftmargin}{15pt}
  }
  \renewcommand{\@listii} {
    \setlength{\itemsep}{5pt}
    \setlength{\topsep}{0pt}
  }
}
\makeatother

%%%%%%%%%%%%%%%%%%%%%%%%%%%%%%% Document Text %%%%%%%%%%%%%%%%%%%%%%%%%%%%%%%%%%

\begin{document}

% note: not using the \maketitle command. We make our own title our own way.
\begin{center}
  {\sffamily \huge \textbf{Chris Clark}} \\ 
  Frederick, MD \enspace\textopenbullet\enspace 
  (571) 732-9827 \enspace\textopenbullet\enspace 
  ludachrispeed@gmail.com
\end{center}

% --------------------------- SECTION: EDUCATION -------------------------------

\vspace{-8pt}
\section*{Education}

  \textsc{University of Virginia}, School of Engineering and Applied Science,
  Charlottesville, VA \\
  \textit{B.S. in Computer Science, May 2011}

  \subsection*{Activities}

  \textit{RideForward}: A UVA-based group of students and professors converting
  gas-powered cars to electric drive. \\
  \textit{Positions held}: Technical team member, business team member, webmaster

  \subsection*{Relevent Coursework}

  \begin{multicols}{2} 
    \raggedright

    \begin{itemize}
    \item \textit{Math}: Calculus I, II, III, Discrete Math, Differential
      Equations, Probability, Linear Algebra
    \item Program and Data Representation (Data Structures)
    \item Advanced Software Development
    \item Defense Against the Dark Arts (computer viruses)
    \item Computer Networks
    \item Operating Systems
    \item Digital Logic Design
    \item Computer Architecture
    \item Theory of Computation (Discrete Math II)
    \item Algorithms
    \item Computer Vision
    \item \textsl{Technical thesis\,}: The Design and Simulation of a Control
      System for a Small Electric Vehicle
    \end{itemize}
  \end{multicols}

% ------------------------ SECTION: RELEVENT SKILLS ----------------------------

  \section*{Relevent Skills}

  \begin{multicols}{2}
    \raggedright

    \subsection*{Web Frameworks and Web Technologies}

    Spring, AngularJS, node.js, Grails, X/HTML, XML, XForms, CSS, Relational
    Databases
    
    \subsection*{Tools/Operating Systems}

    Emacs, Eclipse, IntelliJ, Subversion, Git, Perforce, Linux, Mac OS X

    \subsection*{Programming Languages}

    \textsl{Expert}: Java \\
    \textsl{Proficient}: Bash, Javascript, SQL, Groovy, \LaTeX \\
    \textsl{Familiarity}: C, \CC, MATLAB/Octave, gnuplot, Scala, Ceylon
    
  \end{multicols}
  
% -------------------------- SECTION: EXPERIENCE -------------------------------

\section*{Experience}

\subsection*{Consultant for Red Hat Consulting}

  \vskip 5pt
  \begin{itemize}

    \item \textit{Website Development at GEICO \hfill Aug 2012 -- Present}

      Enterprise web development using mostly Java EE6, Spring MVC, AngularJS,
      and relational databases. Work focused on RESTful web services consumed by
      mobile devices. Also helped to develop some internal metrics-tracking
      applications using Grails and AngularJS.

      Technologies: Java EE6, Spring MVC, JAXB, XForms, MyBatis, Grails,
      AngularJS

%  maybe one day I'll use these sweet tags? Still needs more work.
%
%      \begin{tikzpicture}[tag/.style={fill=blue!20,draw=blue!80,rounded corners}]
%        \node (a) [tag] {Java EE6};
%        \node (b) [tag, right=of a] {Spring};
%        \node (c) [tag, right=of b] {JAXB};
%        \node (d) [tag, right=of c] {XForms};
%        \node (e) [tag, right=of d] {JAXB};
%        \node (f) [tag, right=of e] {MyBatis};
%        \node (g) [tag, right=of f] {Grails};
%        \node (h) [tag, right=of g] {AngularJS};
%      \end{tikzpicture}

    \item \textit{Website development at Cigna \hfill Nov 2011 -- July 2012}

      Member of a two-person team working on an internal application for viewing
      customer information. Also introduced and presented programming best
      practices to other software developers.

      Technologies: Java EE6, JBoss Drools, Spring Framework, iBATIS

    \item \textit{Website development at Kohl's \hfill Aug 2011 -- Nov 2011}

      Took over development on a website meant for use on kiosk machines
      (i.e. touch computers in shopping malls). Ported new functionality from
      the desktop website to the kiosk website. Implemented new UI features as
      requested by a UI requirements team.

      Technologies: Java EE4, Starteam, IntelliJ, JSP

  \end{itemize}

% remove something else, put this in
%  \subsection*{Open Source Projects}

%  I enjoy spending some of my free time working on OpenHAB, the Open Home
%  Automation Bus. It is an application that is intended to make various home
%  automation technologies work together.


% ---------------------- SECTION: OTHER SKILLS/HOBBIES -------------------------

\vskip 4pt
\section*{Other Skills/Hobbies}
\vskip 5pt

  \begin{itemize}
    \item Foreign Languages: Latin (5 years), German (intermediate)
    \item Music: I enjoy playing the cello. I hope to learn the piano.
    \item UVA Varsity Cross Country, Track \& Field \hfill \textit{2006--2008}
  \end{itemize}

\end{document}
